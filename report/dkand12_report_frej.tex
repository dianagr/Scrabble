\documentclass[a4paper, 12pt]{report}


\begin{document}

\title{Scrabble solver}
\date{March 5, 2012}
\author{Frej Connolly \\ connolly@kth.se
        \and Diana Gren \\ dianagr@kth.se}

\maketitle
\tableofcontents


\chapter{Introduction}

\chapter{Problem statement}
\chapter{Background}
\chapter{Approach}
\subsection{DAWG}
When playing Scrabble in a tournament each player needs to finish all the rounds within a time frame, usually 25 minutes. Which means searching for words needs to be fast. Appel and Jacobson showed that when using a directed acyclic word graph DAWG it's easy to find legit words. The data structure is an evolution of a trie where each word corresponds to a path from the root. Each edge representing a letter and nodes flags end of a word. At the time discovering the DAWG disk space were limited and an English dictionary with 94240 words would take up 117150 nodes and 179618 edges. It would occupy over half a megabyte. By using a directed graph instead and letting nodes be shared by several edges it would dramatically reduce size to 19853 nodes and disk space 175 kB. For example the words awesome, awesomeness, awesomely would all share the first 7 nodes. When adding the word greatly it could share the last two letters l and y with the same nodes as the last two for awesomely. Nodes mark the the end of words with a boolean EOW.

\chapter{Results}
\chapter{Conclusions}
\chapter{Discussion}

\end{document}
